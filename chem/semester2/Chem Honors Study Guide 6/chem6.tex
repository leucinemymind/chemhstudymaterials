\documentclass[a4paper, 12pt]{article}
\usepackage{graphicx} % Required for inserting images
\usepackage{textcomp}
\usepackage{fullpage}
\usepackage{amsmath}
\usepackage{xcolor}
\usepackage{float}
\usepackage{geometry}
\usepackage{biblatex}
\geometry{margin=1in}
\usepackage{enumitem}
\usepackage{hyperref}
\usepackage{microtype}
\usepackage{gensymb}
\usepackage{parskip}
\usepackage{tikz}
\usepackage{caption}
\usepackage{cancel}
\usepackage{nicefrac}
\hypersetup{
    colorlinks=true,        % Enable colored links
    linkcolor=teal,         % Set color for internal links
    citecolor=teal,         % Set color for citations
    filecolor=teal,         % Set color for file links
    urlcolor=teal           % Set color for URLs
}

\usepackage[version=4]{mhchem}

\newcommand{\degC}{$\degree$C \,}
\newcommand{\degF}{$\degree$F \,}
\newcommand{\R}{\left(0.0821 \: \frac{L \cdot atm}{mol \cdot \text{K}}\right)}
\newcommand{\cunits}{$\frac{J}{g \degree \text{C}}$}
\newcommand{\Hf}{$\Delta H_\text{f}$} % heat of formation
\newcommand{\mathHf}{\Delta H_\text{f}} %heat of formation in math mode

\title{Chem Honors Study Guide 6}
\author{Test 3 S2}
\date{Test date: TBD}

\begin{document}

\maketitle

\section{Gasses and Heat in Stoichiometry}

\subsection*{Gasses}
$$6 \text{HCl} + 2 \text{Al} \longrightarrow 2 \text{AlCl}_3 + 3\text{H}_2$$

At STP, how many $ml$ of H$_2$ gas are produced from 12 $g$ of solid Al? (1 $mol$ = 22.4 $L$ at STP)

Using stoichiometry:

$$(12 \: g \: \text{Al}) \times \left(\frac{1 \: mol \: \text{Al}}{26.98 \: g \: \text{Al}}\right) \times \left(\frac{3 \: mol \: \text{H}_2}{2 \: mol \: \text{Al}}\right) \times \left(\frac{22.4 \: L \: \text{H}_2}{1 \: mol \: \text{H}_2}\right) \times \left(\frac{1000 \: ml}{1 \: L}\right)$$
$$ = \boxed{14944.4 \: ml}$$

\subsection*{Heats of Formation}
\Hf{} is the heat absorbed/released when compounds are formed from elemental units. The \Hf{} of elements, including diatomic elements, is always 0.

Heats of formation equation:

\begin{equation} \label{heatsofform}
\Delta H_\text{rxn} = \sum \Delta H_\text{f(products)} -  \Delta H_\text{f(reactants)}
\end{equation}

$$\text{CS}_2 + 3\text{O}_2 \longrightarrow \text{CO}_2 + 2\text{SO}_2$$

Find the heat of formation given the following: 

$$\mathHf \: (\text{CO}_2) = -393.5 \: \frac{kJ}{mol}$$
$$\mathHf \: (\text{SO}_2) = -296.8 \: \frac{kJ}{mol}$$
$$\mathHf \: (\text{CS}_2) = 87.9 \: \frac{kJ}{mol}$$

Solution: Using \ref{heatsofform}:

$$ [-393.5 + 2(-296.8)] - [3(0) + 87.9] $$
$$ = \boxed{1075 \: \frac{kJ}{mol}} $$

\end{document}